%%%%%%%%%%%%%%%%%%%%%%%%%%%%%%
% LATEX-TEMPLATE TECHNISCH RAPPORT
%-------------------------------------------------------------------------------
% This template is derived from the AVI PT report template.
%%%%%%%%%%%%%%%%%%%%%%%%%%%%%%

%-------------------------------------------------------------------------------
%	PACKAGES EN DOCUMENT CONFIGURATIE
%-------------------------------------------------------------------------------

\documentclass{uva-inf-article}
%\usepackage[dutch]{babel}
%\usepackage{csquotes}

\usepackage[style=numeric-comp]{biblatex}
\addbibresource{PT-essay.bib}

%-------------------------------------------------------------------------------
%	GEGEVENS VOOR IN DE TITEL
%-------------------------------------------------------------------------------

% Vul de naam van de opdracht in.
\assignment{Individual Written Assessment}
% Vul het soort opdracht in.
\assignmenttype{Essay}
% Vul de titel van de eindopdracht in.
\title{Designing \texttt{legalease}, a Smart Contracts programming language}

% Vul de volledige namen van alle auteurs in.
\author{René Kok}
% Vul de corresponderende UvAnetID's in.
\uvanetid{13671146}

% Vul eventueel ook de naam van de docent of vakcoordinator toe.
\docent{dr. Ana Oprescu}
\course{Programmeertalen}
% Te vinden op onder andere Datanose.
\courseid{}

% Dit is de datum die op het document komt te staan. Standaard is dat vandaag.
\date{\today}

%-------------------------------------------------------------------------------
%	VOORPAGINA
%-------------------------------------------------------------------------------

\begin{document}
\maketitle

%-------------------------------------------------------------------------------
%	INHOUDSOPGAVE EN ABSTRACT
%-------------------------------------------------------------------------------

% Niet doen bij korte verslagen en rapporten
%\tableofcontents
%\begin{abstract}
%\lipsum[13]
%\end{abstract}

%-------------------------------------------------------------------------------
%	INTRODUCTIE
%-------------------------------------------------------------------------------

\section{Introduction}
Smart Contracts \cite{relevant-citation-1} are key to ...
%\lipsum[1]

\subsection{Relevant characteristics of application domain}
%\lipsum[2]

\subsection{Use case description}
%\lipsum[3]

%-------------------------------------------------------------------------------
%	METHODE
%-------------------------------------------------------------------------------

\section{Analysis}
%\lipsum[3]
\subsection{The type system}
\subsubsection{Type checking}
Because smart contracts are designed for transactions, you want a smart contract to be accurate and secure before executing the smart contract. 
That is why I chose to have the language static typed.
Static type checking has the advantages that the smart contract is guaranteed to meet a number of type safety features for all possible inputs.
In addition, a static typing language is better optimized as opposed to dynamically typed language, because the compiler knows if a program is correctly typed.
This results in a smaller and faster binary because no dynamic safety checks need to be performed.
\subsection{State management}
\subsection{Compilation/interpretation strategy}
\subsection{Evaluation strategy (lazy/eager)}
\subsection{Parameter evaluation strategy (call by value/reference)}
\subsection{Communication semantics (synchronous/asynchronous)}
\subsection{Higher-order functions}
\subsection{Anonymous functions}

\section{Discussion}
%\lipsum[5]

\section{Conclusions}
%\lipsum[7]

%-------------------------------------------------------------------------------
%	REFERENTIES
%-------------------------------------------------------------------------------

\printbibliography

%-------------------------------------------------------------------------------
%	BIJLAGEN EN EINDE
%-------------------------------------------------------------------------------

%\section{Bijlage A}
%\section{Bijlage B}
%\section{Bijlage C}
\end{document}
